\chapter{The Symmetry View}

\begin{goals}
\begin{itemize}
    \item Understand geometric deep learning
    \item See how symmetry constrains architecture
    \item Know what this framework can and cannot do
\end{itemize}
\end{goals}

\section{The Core Idea}

Many tasks have symmetries. If the architecture respects them, learning is easier.

\begin{example}
Image classification:
\begin{itemize}
    \item A cat in the corner is still a cat
    \item The task is \emph{translation invariant}
    \item CNN kernels slide across positions $\to$ translation equivariant
\end{itemize}
\end{example}

\begin{definition}[Invariance and Equivariance]
Let $G$ be a group acting on input space $X$ and output space $Y$.
\begin{itemize}
    \item $f: X \to Y$ is \textbf{$G$-invariant} if $f(g \cdot x) = f(x)$ for all $g \in G$
    \item $f: X \to Y$ is \textbf{$G$-equivariant} if $f(g \cdot x) = g \cdot f(x)$ for all $g \in G$
\end{itemize}
\end{definition}

Invariance: output doesn't change. Equivariance: output transforms the same way as input.

\section{Geometric Deep Learning}

Bronstein et al. unified many architectures under one framework:

\begin{center}
\begin{tabular}{lll}
\textbf{Domain} & \textbf{Symmetry Group} & \textbf{Architecture} \\
\hline
Images & Translations $\mathbb{Z}^2$ & CNN \\
Sets & Permutations $S_n$ & DeepSets \\
Graphs & Node permutations & GNN \\
Spheres & Rotations $SO(3)$ & Spherical CNN \\
Manifolds & Gauge transformations & Gauge CNN \\
\end{tabular}
\end{center}

\begin{keyinsight}
Once you identify the symmetry group, the architecture follows.

The ``right'' architecture is the one that is equivariant to the task's symmetry.
\end{keyinsight}

\section{Why Equivariance Helps}

\subsection{Parameter Efficiency}

Equivariant layers have fewer free parameters.

\begin{example}
For $n \times n$ images:
\begin{itemize}
    \item Fully connected: $n^4$ parameters
    \item Convolutional ($k \times k$ kernel): $k^2$ parameters
\end{itemize}
The constraint of translation equivariance forces weight sharing.
\end{example}

\subsection{Data Efficiency}

You don't need to learn what symmetry gives you for free.

\begin{intuition}
A non-equivariant network must learn separately:
\begin{itemize}
    \item ``Cat in top-left is a cat''
    \item ``Cat in top-right is a cat''
    \item ``Cat in bottom-left is a cat''
    \item ...
\end{itemize}
An equivariant network learns once: ``This pattern is a cat.''
\end{intuition}

\subsection{Generalization}

Equivariance is an inductive bias that matches the task structure.

Recent work (2025) confirms: equivariant models
\begin{enumerate}
    \item Are more data efficient
    \item Outpace non-equivariant models as task complexity increases
    \item Can benefit from targeted symmetry breaking when data isn't perfectly symmetric
\end{enumerate}

\section{The Mathematical Framework}

\subsection{Group Representations}

A representation of group $G$ on vector space $V$ is a homomorphism:
\[
\rho: G \to GL(V)
\]

For neural networks, we need $G$ to act on both input and output spaces.

\subsection{Equivariant Linear Maps}

\begin{theorem}
A linear map $\phi: V \to W$ is $G$-equivariant iff:
\[
\phi \circ \rho_V(g) = \rho_W(g) \circ \phi \quad \forall g \in G
\]
\end{theorem}

The space of such maps is determined by representation theory — specifically, by how $\rho_V$ and $\rho_W$ decompose into irreducibles.

\subsection{Building Equivariant Networks}

\begin{enumerate}
    \item Choose input/output representations
    \item Compute the space of equivariant linear maps (using Schur's lemma)
    \item Add equivariant nonlinearities
    \item Stack layers
\end{enumerate}

The ``Equivariant MLP'' algorithm automates this for any matrix group.

\section{Limitations}

\subsection{Requires Known Symmetry}

\begin{warning}
Geometric deep learning assumes you \emph{know} the symmetry group upfront.

But what if:
\begin{itemize}
    \item The symmetry is unknown?
    \item The symmetry is approximate?
    \item There is no group symmetry, but some other structure?
\end{itemize}
\end{warning}

\subsection{Only Handles Group Symmetry}

Not all structure is group-theoretic.

\begin{example}
Consider learning a DFA (deterministic finite automaton):
\begin{itemize}
    \item The ``structure'' is the transition function
    \item There may be no natural group action
    \item Geometric DL doesn't directly apply
\end{itemize}
\end{example}

\subsection{Expressiveness vs. Bias Tradeoff}

Making a network equivariant restricts its function class.

\begin{theorem}[Universal Approximation]
$G$-equivariant networks can approximate any $G$-equivariant function.

But they \emph{cannot} approximate non-equivariant functions.
\end{theorem}

If your task isn't actually $G$-symmetric, an equivariant network will fail.

\section{Relevance to Our Question}

Geometric deep learning answers: \emph{given symmetry group $G$, what architecture?}

But we want to answer: \emph{given semantic target (e.g., a coalgebra), what architecture?}

These overlap when the semantics has group symmetry. But many structures (automata, Kripke frames, arbitrary coalgebras) don't have obvious group structure.

\begin{keyinsight}
Symmetry is one source of structure. But it's not the only one.

We need a more general theory that handles:
\begin{itemize}
    \item Group symmetry (Geometric DL)
    \item Algebraic structure (semirings, monoids)
    \item Coalgebraic structure (transition systems, behaviors)
\end{itemize}
\end{keyinsight}
