\chapter{The Optimization Geometry View}

\begin{goals}
\begin{itemize}
    \item See parameter space as a Riemannian manifold
    \item Understand Fisher information as a metric
    \item Learn why some parameterizations optimize better than others
\end{itemize}
\end{goals}

\section{The Core Idea}

Not all directions in parameter space are equal.

\begin{intuition}
Imagine optimizing on an ellipse versus a circle.
\begin{itemize}
    \item Circle: all directions make equal progress
    \item Elongated ellipse: some directions are ``steep,'' others ``flat''
\end{itemize}
Gradient descent struggles with elongated landscapes — it zigzags.
\end{intuition}

The \emph{shape} of the loss landscape affects optimization difficulty. This shape depends on how we parameterize.

\section{Fisher Information}

\subsection{Definition}

\begin{definition}[Fisher Information Matrix]
For a parameterized distribution $p_\theta(x)$:
\[
F_{ij}(\theta) = \mathbb{E}_{x \sim p_\theta} \left[ \frac{\partial \log p_\theta(x)}{\partial \theta_i} \cdot \frac{\partial \log p_\theta(x)}{\partial \theta_j} \right]
\]
\end{definition}

Equivalently:
\[
F(\theta) = -\mathbb{E}\left[ \frac{\partial^2 \log p_\theta(x)}{\partial \theta_i \partial \theta_j} \right]
\]

\begin{intuition}
$F(\theta)$ measures sensitivity: how much does the output distribution change when we perturb $\theta$?
\begin{itemize}
    \item Large eigenvalue: sensitive direction
    \item Small eigenvalue: insensitive direction
    \item Zero eigenvalue: redundant direction (multiple $\theta$ give same output)
\end{itemize}
\end{intuition}

\subsection{Fisher as Riemannian Metric}

The Fisher information defines a Riemannian metric on parameter space:
\[
ds^2 = \sum_{ij} F_{ij}(\theta) \, d\theta_i \, d\theta_j
\]

This is the \textbf{Fisher-Rao metric}. It measures distance between nearby distributions, not between parameter values.

\begin{keyinsight}
The Fisher metric is \emph{intrinsic}: it depends only on the family of distributions, not on how we parameterize it.

Different parameterizations give different coordinate representations of the same geometry.
\end{keyinsight}

\section{Natural Gradient}

\subsection{Amari's Insight}

Ordinary gradient descent moves in the steepest direction in \emph{parameter space}.

But we care about \emph{distribution space}. The steepest direction there is:
\[
\tilde{\nabla} L = F^{-1} \nabla L
\]

This is the \textbf{natural gradient} (Amari, 1998).

\begin{definition}[Natural Gradient Descent]
\[
\theta_{t+1} = \theta_t - \eta \, F(\theta_t)^{-1} \nabla L(\theta_t)
\]
\end{definition}

\subsection{Properties}

\begin{enumerate}
    \item \textbf{Parameterization invariant}: Same trajectory regardless of how we parameterize
    \item \textbf{Fisher efficient}: Asymptotically optimal convergence rate
    \item \textbf{Corrects for curvature}: No zigzagging on elongated landscapes
\end{enumerate}

\subsection{The Catch}

\begin{warning}
Computing $F^{-1}$ is expensive: $O(n^3)$ for $n$ parameters.

For neural networks with millions of parameters, this is infeasible.
\end{warning}

Approximations:
\begin{itemize}
    \item \textbf{Diagonal Fisher}: Keep only diagonal entries
    \item \textbf{K-FAC}: Kronecker-factored approximation (Martens \& Grosse, 2015)
    \item \textbf{Block diagonal}: One block per layer
\end{itemize}

\section{Condition Number}

\subsection{Definition}

\begin{definition}
The \emph{condition number} of $F$ is:
\[
\kappa(F) = \frac{\lambda_{\max}(F)}{\lambda_{\min}(F)}
\]
\end{definition}

\begin{center}
\begin{tabular}{ll}
$\kappa \approx 1$ & Easy to optimize (isotropic) \\
$\kappa \gg 1$ & Hard to optimize (elongated) \\
$\kappa = \infty$ & Redundant parameterization (singular $F$) \\
\end{tabular}
\end{center}

\subsection{Redundancy and Symmetry}

If multiple parameters give the same function, $F$ has zero eigenvalues.

\begin{example}
A fully connected network learning a translation-invariant function:
\begin{itemize}
    \item Many weight configurations give the same input-output map
    \item $F$ is singular along these ``symmetry directions''
    \item $\kappa = \infty$
\end{itemize}
\end{example}

\begin{keyinsight}
Symmetry in the target function, not reflected in the parameterization, causes singular Fisher information and poor conditioning.
\end{keyinsight}

\section{What Geometry Tells Us}

\subsection{Strengths}

\begin{itemize}
    \item \textbf{Quantitative}: Condition number is measurable
    \item \textbf{Explains difficulty}: High $\kappa$ $\to$ slow convergence
    \item \textbf{Suggests fix}: Natural gradient, or better parameterization
\end{itemize}

\subsection{Limitations}

\begin{itemize}
    \item \textbf{Local}: $F(\theta)$ depends on current $\theta$, not global landscape
    \item \textbf{Expensive}: Natural gradient is costly to compute
    \item \textbf{Doesn't prescribe architecture}: Says ``reduce $\kappa$,'' not how
\end{itemize}

\section{Relevance to Our Question}

The geometry view suggests:

\begin{enumerate}
    \item Good parameterization $\to$ low condition number
    \item If target has symmetry group $G$, parameterization should respect $G$
    \item Otherwise, $F$ will be singular
\end{enumerate}

This motivates studying equivariant parameterizations (next chapter).

But geometry alone doesn't tell us: given an arbitrary target (not just symmetric functions), what's the best parameterization?
