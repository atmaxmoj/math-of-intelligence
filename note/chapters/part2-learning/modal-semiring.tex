\chapter{Modal Operators in Semirings}

\begin{goals}
\begin{itemize}
    \item Face the core question: how to interpret $\necessary$/$\possible$ over semirings
    \item Survey existing approaches: fuzzy, many-valued, residuated lattices
    \item Understand what's solved and what's open
    \item Know the implications for learning
\end{itemize}
\end{goals}

\section{The Problem}

In classical modal logic:
\[
\mathcal{M}, w \models \necessary\varphi \iff \forall v: wRv \Rightarrow \mathcal{M}, v \models \varphi
\]

This involves:
\begin{itemize}
    \item $\forall$ — universal quantification (AND over all successors)
    \item $\Rightarrow$ — implication (if accessible, then...)
\end{itemize}

In semiring-valued setting:
\begin{itemize}
    \item $R(w,v) \in S$ — soft accessibility
    \item $\sem{\varphi}_v \in S$ — soft truth value
\end{itemize}

\textbf{Question}: What is $\sem{\necessary\varphi}_w$?

\section{Existing Work: Three Traditions}

This is \emph{not} a completely open problem. There's substantial literature.

\subsection{Modal Semirings (Algebraic)}

Möller, Desharnais, and others developed \emph{modal semirings} as an algebraic axiomatization.

\begin{itemize}
    \item Define \emph{domain} operation: $\mathrm{dom}(x) = |x\rangle 1$
    \item Define box via de Morgan: $|x]p = \neg|x\rangle\neg p$
    \item Focus on algebraic laws, not many-valued truth
\end{itemize}

Key reference: ``Some Uses of Modal Semirings'' (Möller \& Desharnais, WADT 2024).

This approach is about \emph{algebraic structure}, not about truth values in $[0,1]$.

\subsection{Many-Valued Modal Logic (Lattice-Valued)}

Bou, Esteva, Godo, and others study modal logic over \emph{residuated lattices}.

\begin{definition}[Lattice-Valued Kripke Model]
Over a residuated lattice $A = \langle A, 0, 1, \land, \lor, \otimes, \to \rangle$:
\begin{itemize}
    \item $R : W \times W \to A$ (many-valued accessibility)
    \item $e : \mathrm{Var} \times W \to A$ (many-valued valuation)
\end{itemize}
\end{definition}

The semantics of $\necessary$:
\[
\sem{\necessary\varphi}_w = \bigwedge_v \left( R(w,v) \to \sem{\varphi}_v \right)
\]
where $\to$ is the residual of $\otimes$.

Key reference: ``On the Minimum Many-Valued Modal Logic over a Finite Residuated Lattice'' (Bou et al., 2009).

\subsection{Fuzzy Modal Logic (T-Norm Based)}

Fuzzy modal logic uses t-norms and their residuals.

\begin{definition}[Fuzzy Modal Semantics]
For a t-norm $\otimes$ with residual $\Rightarrow$:
\begin{align*}
\sem{\necessary\varphi}_w &= \inf_v \left( R(w,v) \Rightarrow \sem{\varphi}_v \right) \\
\sem{\possible\varphi}_w &= \sup_v \left( R(w,v) \otimes \sem{\varphi}_v \right)
\end{align*}
\end{definition}

Common choices:
\begin{center}
\begin{tabular}{lll}
\textbf{Name} & $a \otimes b$ & $a \Rightarrow b$ \\
\hline
Gödel & $\min(a,b)$ & $\begin{cases} 1 & a \leq b \\ b & a > b \end{cases}$ \\
Łukasiewicz & $\max(0, a+b-1)$ & $\min(1, 1-a+b)$ \\
Product & $a \cdot b$ & $\begin{cases} 1 & a \leq b \\ b/a & a > b \end{cases}$ \\
\end{tabular}
\end{center}

\section{What's Known}

\begin{theorem}[Failure of Axiom K]
In many-valued modal logic with non-Boolean accessibility, the axiom
\[
\necessary(\varphi \to \psi) \to (\necessary\varphi \to \necessary\psi)
\]
generally \textbf{fails}.
\end{theorem}

\begin{theorem}[Non-Interdefinability]
In general, $\possible\varphi \not\equiv \neg\necessary\neg\varphi$ when $R$ is many-valued.
\end{theorem}

\begin{center}
\begin{tabular}{ll}
\textbf{Property} & \textbf{Status} \\
\hline
Semantics for $\necessary$/$\possible$ & Defined (inf/sup with residual) \\
Axiom K & Fails in general \\
$\necessary$/$\possible$ duality & Fails in general \\
Completeness & Case-by-case, often unknown \\
Decidability & Case-by-case \\
\end{tabular}
\end{center}

\section{What's Still Open}

\begin{enumerate}
    \item \textbf{General semirings}: Most work uses residuated lattices (which have a specific implication $\to$). Arbitrary semirings don't have a canonical implication.

    \item \textbf{Tropical semiring}: $(\mathbb{R} \cup \{\infty\}, \min, +)$ — what's the right modal semantics? Not well-studied.

    \item \textbf{Gradient semiring}: What happens when we track derivatives through modal operators? Unexplored.

    \item \textbf{Best semantics for learning}: Which choice of t-norm/residual leads to best optimization landscape? No one has studied this.

    \item \textbf{Convergence}: When does soft modal semantics converge to crisp? Conditions unclear.
\end{enumerate}

\section{For Learning: Practical Choices}

Given the theory, what should we use for gradient-based learning?

\begin{warning}
$\inf$ and $\sup$ have zero gradients almost everywhere. Not suitable for learning.
\end{warning}

\subsection{Option 1: Soft Inf/Sup}

Replace $\inf$ with softmin:
\[
\mathrm{softmin}_\tau(x_1, \ldots, x_n) = -\tau \log \sum_i e^{-x_i/\tau}
\]

As $\tau \to 0$, this approaches true $\min$.

\subsection{Option 2: Product Semantics}

Use product t-norm throughout:
\begin{align*}
\sem{\necessary\varphi}_w &= \prod_v \left( 1 - R(w,v) \cdot (1 - \sem{\varphi}_v) \right) \\
\sem{\possible\varphi}_w &= 1 - \prod_v \left( 1 - R(w,v) \cdot \sem{\varphi}_v \right)
\end{align*}

Fully differentiable, gradients everywhere.

\subsection{Option 3: Weighted Average}

For $\necessary$ as ``expected truth over successors'':
\[
\sem{\necessary\varphi}_w = \frac{\sum_v R(w,v) \cdot \sem{\varphi}_v}{\sum_v R(w,v)}
\]

Simple, differentiable, but doesn't match classical semantics.

\section{Recommendation}

\begin{keyinsight}
For learning:
\begin{enumerate}
    \item Use \textbf{product t-norm} or \textbf{softmin/softmax}
    \item Accept that axiom K may fail during training
    \item Add regularization to push toward crisp values
    \item After convergence, verify classical properties
\end{enumerate}
\end{keyinsight}

\section{Summary}

\begin{center}
\begin{tabular}{lll}
\textbf{Aspect} & \textbf{Status} & \textbf{Reference} \\
\hline
Algebraic axioms & Solved & Möller et al. \\
Residuated lattice semantics & Solved & Bou et al. \\
Fuzzy/t-norm semantics & Solved & Hájek, Fitting \\
Arbitrary semirings & Partially open & --- \\
Differentiable semantics & Open & --- \\
Best choice for learning & Open & --- \\
\end{tabular}
\end{center}

The mathematics of many-valued modal logic is well-developed. The question of which variant works best for gradient-based learning is new territory.
