\chapter{Differentiable Graded Modalities}

\begin{goals}
\begin{itemize}
    \item Combine graded quantifiers with differentiable learning
    \item See how to soften counting modalities
    \item Understand this as a potential novel contribution
\end{itemize}
\end{goals}

\section{The Gap}

From Part I: graded modalities $\possible_{\geq n}$, $\mathsf{M}$, $P_{\geq r}$ are expressive.

From Part II: semiring-valued coalgebra enables learning.

But existing work doesn't combine them:

\begin{center}
\begin{tabular}{llll}
\textbf{Work} & \textbf{Learns $R$?} & \textbf{Graded?} & \textbf{Differentiable?} \\
\hline
MLNNs (2024) & \checkmark & $\times$ (only $\necessary$/$\possible$) & \checkmark \\
Graded Modal & $\times$ & \checkmark & $\times$ \\
This chapter & \checkmark & \checkmark & \checkmark \\
\end{tabular}
\end{center}

\section{Softening Graded Modalities}

Recall the crisp semantics:
\[
\mathcal{M}, w \models \possible_{\geq n}\varphi \iff |\{v : wRv \land v \models \varphi\}| \geq n
\]

This is a hard threshold. Not differentiable.

\begin{definition}[Soft Graded Modality]
For fuzzy $R : W \times W \to [0,1]$ and $\sem{\varphi}_v \in [0,1]$:
\[
\sem{\possible_{\geq n}\varphi}_w = \sigma\left( \sum_v R(w,v) \cdot \sem{\varphi}_v - n + \frac{1}{2} \right)
\]
where $\sigma$ is the sigmoid function.
\end{definition}

\begin{intuition}
The sum $\sum_v R(w,v) \cdot \sem{\varphi}_v$ is a soft count of how many successors satisfy $\varphi$. The sigmoid turns ``soft count $\geq n$'' into a value in $[0,1]$.
\end{intuition}

\section{Other Soft Quantifiers}

\begin{center}
\begin{tabular}{ll}
\textbf{Quantifier} & \textbf{Soft Version} \\
\hline
$\possible_{\geq n}\varphi$ & $\sigma(\sum_v R \cdot \sem{\varphi}_v - n + 0.5)$ \\
$\possible_{\leq n}\varphi$ & $\sigma(n + 0.5 - \sum_v R \cdot \sem{\varphi}_v)$ \\
$\mathsf{M}\varphi$ & $\sigma(\sum_v R \cdot \sem{\varphi}_v - \sum_v R \cdot \sem{\neg\varphi}_v)$ \\
$P_{\geq r}\varphi$ & $\sigma(\frac{\sum_v R \cdot \sem{\varphi}_v}{\sum_v R(w,v)} - r)$ \\
\end{tabular}
\end{center}

\section{Learning with Graded Constraints}

Now we can train:

\begin{example}[Robust Planning]
``Learn a transition system where every state has at least 2 safe successors.''

Constraint: $\forall w. \possible_{\geq 2}\mathsf{safe}$

Loss: $L = \sum_w (1 - \sem{\possible_{\geq 2}\mathsf{safe}}_w)^2$

Optimize: gradient descent on $R$ (soft accessibility).
\end{example}

\begin{example}[Probabilistic Requirement]
``With probability $\geq 0.8$, action leads to goal.''

Constraint: $P_{\geq 0.8}\mathsf{goal}$

Loss derived from soft semantics, optimize $R$.
\end{example}

\section{Convergence}

\begin{conjecture}
With appropriate regularization (entropy penalty, temperature annealing), soft graded modalities converge to crisp:
\begin{itemize}
    \item $R(w,v) \to \{0, 1\}$
    \item $\sem{\possible_{\geq n}\varphi}_w \to \{0, 1\}$
\end{itemize}
The learned structure is a classical Kripke model satisfying the graded constraints.
\end{conjecture}

\section{Open Questions}

\begin{enumerate}
    \item What are the best soft approximations? (Sigmoid? Softplus?)
    \item Convergence guarantees?
    \item How do meta-theoretic properties (decidability, completeness) transfer?
    \item Approximation bounds: how close is soft to crisp?
\end{enumerate}

\begin{keyinsight}
Differentiable graded modalities let us \emph{learn} structures satisfying counting constraints. This combines the expressiveness of graded modal logic with the learnability of neural networks.
\end{keyinsight}
