\chapter{The Gradient Semiring}

\begin{goals}
\begin{itemize}
    \item Understand automatic differentiation algebraically
    \item See how gradients flow through semiring operations
    \item Appreciate: inference and learning in one pass
\end{itemize}
\end{goals}

\section{The Problem}

We want to optimize over semiring computations.

Given a semiring expression (e.g., probabilistic inference), we need gradients w.r.t. parameters.

Na\"ive approach: compute forward, then backpropagate.

Better approach: \textbf{embed gradients into the semiring itself}.

\section{Dual Numbers}

\begin{definition}[Dual Numbers]
A \emph{dual number} is $a + b\epsilon$ where $\epsilon^2 = 0$.

Arithmetic:
\begin{align*}
(a + b\epsilon) + (c + d\epsilon) &= (a+c) + (b+d)\epsilon \\
(a + b\epsilon) \times (c + d\epsilon) &= ac + (ad + bc)\epsilon
\end{align*}
\end{definition}

\begin{keyinsight}
If $f(x) = a + b\epsilon$ when we plug in $x = x_0 + 1 \cdot \epsilon$, then:
\begin{itemize}
    \item $a = f(x_0)$ (the value)
    \item $b = f'(x_0)$ (the derivative!)
\end{itemize}
\end{keyinsight}

\section{The Gradient Semiring}

\begin{definition}[Gradient Semiring]
Elements: pairs $(v, g)$ where $v$ = value, $g$ = gradient.

Operations:
\begin{align*}
(v_1, g_1) \oplus (v_2, g_2) &= (v_1 + v_2, \; g_1 + g_2) \\
(v_1, g_1) \otimes (v_2, g_2) &= (v_1 \cdot v_2, \; v_1 \cdot g_2 + v_2 \cdot g_1)
\end{align*}

Identities: $\mathbf{0} = (0, 0)$, $\mathbf{1} = (1, 0)$.
\end{definition}

\begin{proposition}
The gradient semiring is a semiring.
\end{proposition}

\begin{proof}
Check the axioms. The key is that $\otimes$ follows the product rule.
\end{proof}

\section{Forward-Mode Automatic Differentiation}

To compute $f(x)$ and $\frac{\partial f}{\partial x}$ simultaneously:
\begin{enumerate}
    \item Replace $x$ with $(x, 1)$ (value $x$, derivative $1$)
    \item Compute using gradient semiring operations
    \item Extract: first component = value, second = derivative
\end{enumerate}

\begin{example}
Compute $f(x) = x^2 + x$ at $x = 3$:
\begin{align*}
x &= (3, 1) \\
x^2 &= (3, 1) \otimes (3, 1) = (9, 6) \\
x^2 + x &= (9, 6) \oplus (3, 1) = (12, 7)
\end{align*}
So $f(3) = 12$ and $f'(3) = 7$. Check: $f'(x) = 2x + 1$, so $f'(3) = 7$. ✓
\end{example}

\section{Application: Differentiable Logic}

Any semiring computation can be made differentiable:
\begin{enumerate}
    \item Write your logic in semiring form
    \item Instantiate with gradient semiring
    \item Get gradients for free
\end{enumerate}

\begin{keyinsight}
This is how DeepProbLog works: probabilistic logic programming with gradients. Inference and learning happen in a single forward pass.
\end{keyinsight}

\section{Limitations}

Forward-mode computes $\frac{\partial f}{\partial x_i}$ for one $x_i$ at a time.

For many parameters, need reverse-mode (backpropagation).

Reverse-mode is also algebraic, but more complex (requires ``transposing'' the computation).
