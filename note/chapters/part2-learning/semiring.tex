\chapter{Semiring: The Abstraction}

\begin{goals}
\begin{itemize}
    \item See the common structure behind fuzzy, probabilistic, tropical
    \item Understand the formal definition of semiring
    \item Appreciate why this abstraction is useful
\end{itemize}
\end{goals}

\section{The Pattern We've Seen}

Recall:

\begin{center}
\begin{tabular}{llllll}
\textbf{Name} & $S$ & $\oplus$ (OR) & $\otimes$ (AND) & $\mathbf{0}$ & $\mathbf{1}$ \\
\hline
Boolean & $\{0,1\}$ & $\lor$ & $\land$ & $0$ & $1$ \\
Fuzzy (Gödel) & $[0,1]$ & $\max$ & $\min$ & $0$ & $1$ \\
Probabilistic & $\mathbb{R}_{\geq 0}$ & $+$ & $\times$ & $0$ & $1$ \\
\end{tabular}
\end{center}

All satisfy the same algebraic laws. Let's name this structure.

\section{Definition}

\begin{definition}[Semiring]
A \emph{semiring} is a tuple $(S, \oplus, \otimes, \mathbf{0}, \mathbf{1})$ where:
\begin{enumerate}
    \item $(S, \oplus, \mathbf{0})$ is a commutative monoid:
    \begin{itemize}
        \item $a \oplus (b \oplus c) = (a \oplus b) \oplus c$
        \item $a \oplus b = b \oplus a$
        \item $a \oplus \mathbf{0} = a$
    \end{itemize}
    \item $(S, \otimes, \mathbf{1})$ is a monoid:
    \begin{itemize}
        \item $a \otimes (b \otimes c) = (a \otimes b) \otimes c$
        \item $a \otimes \mathbf{1} = \mathbf{1} \otimes a = a$
    \end{itemize}
    \item $\otimes$ distributes over $\oplus$:
    \begin{itemize}
        \item $a \otimes (b \oplus c) = (a \otimes b) \oplus (a \otimes c)$
        \item $(a \oplus b) \otimes c = (a \otimes c) \oplus (b \otimes c)$
    \end{itemize}
    \item $\mathbf{0}$ annihilates:
    \begin{itemize}
        \item $a \otimes \mathbf{0} = \mathbf{0} \otimes a = \mathbf{0}$
    \end{itemize}
\end{enumerate}
\end{definition}

\begin{intuition}
$\oplus$ = ``or'' / ``choice'' / ``combine alternatives''

$\otimes$ = ``and'' / ``sequence'' / ``combine steps''

$\mathbf{0}$ = ``impossible'' / ``failure''

$\mathbf{1}$ = ``trivially true'' / ``do nothing''
\end{intuition}

\section{More Examples}

\begin{definition}[Tropical Semiring]
$(\mathbb{R} \cup \{+\infty\}, \min, +, +\infty, 0)$

\begin{itemize}
    \item $a \oplus b = \min(a, b)$: choose the better option
    \item $a \otimes b = a + b$: accumulate costs
    \item $\mathbf{0} = +\infty$: infinite cost = impossible
    \item $\mathbf{1} = 0$: zero cost = free
\end{itemize}
\end{definition}

\begin{keyinsight}
In the tropical semiring, logical inference becomes optimization!

``Find a satisfying assignment'' becomes ``find the minimum-cost path.''
\end{keyinsight}

\begin{definition}[Viterbi Semiring]
$([0,1], \max, \times, 0, 1)$

Used in HMMs: find the most probable path.
\end{definition}

\begin{definition}[Log Semiring]
$(\mathbb{R} \cup \{-\infty\}, \mathrm{logsumexp}, +, -\infty, 0)$

Numerically stable version of probabilistic semiring.
\end{definition}

\section{Why Semirings?}

\begin{enumerate}
    \item \textbf{Genericity}: Write algorithm once, instantiate for different semirings
    \item \textbf{Correctness}: Algebraic laws guarantee properties
    \item \textbf{Modularity}: Change interpretation without changing structure
\end{enumerate}

\begin{example}
The same dynamic programming algorithm:
\begin{itemize}
    \item Boolean semiring → reachability (is there a path?)
    \item Tropical semiring → shortest path (what's the minimum cost?)
    \item Probabilistic semiring → most probable path
    \item Counting semiring $(\mathbb{N}, +, \times, 0, 1)$ → count paths
\end{itemize}
\end{example}
