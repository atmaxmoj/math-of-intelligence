\chapter{Completeness}

\begin{goals}
\begin{itemize}
    \item Understand normal modal logics and their axiomatizations
    \item Master the canonical model construction
    \item Prove completeness for K, T, S4, S5
    \item Understand what canonicity means and why it matters
\end{itemize}
\end{goals}

\section{Normal Modal Logics}

\begin{definition}[Normal Modal Logic]
A \emph{normal modal logic} is a set $L$ of formulas containing:
\begin{itemize}
    \item All propositional tautologies
    \item The \textbf{K axiom}: $\necessary(p \to q) \to (\necessary p \to \necessary q)$
\end{itemize}
and closed under:
\begin{itemize}
    \item Modus ponens: from $\varphi$ and $\varphi \to \psi$, derive $\psi$
    \item Necessitation: from $\varphi$, derive $\necessary\varphi$
\end{itemize}
\end{definition}

\begin{example}
\begin{itemize}
    \item \textbf{K} = minimal normal modal logic
    \item \textbf{T} = K + $\necessary p \to p$ (reflexivity)
    \item \textbf{S4} = T + $\necessary p \to \necessary\necessary p$ (transitivity)
    \item \textbf{S5} = S4 + $p \to \necessary\possible p$ (symmetry)
\end{itemize}
\end{example}

\section{Maximal Consistent Sets}

\begin{definition}[Maximal Consistent Set]
A set $\Gamma$ of formulas is \emph{maximally $L$-consistent} if:
\begin{itemize}
    \item $\Gamma$ is $L$-consistent (no derivation of $\bot$)
    \item $\Gamma$ is maximal: for all $\varphi$, either $\varphi \in \Gamma$ or $\neg\varphi \in \Gamma$
\end{itemize}
\end{definition}

\begin{lemma}[Lindenbaum]
Every consistent set can be extended to a maximally consistent set.
\end{lemma}

\begin{proof}
Enumerate all formulas. Add each one if consistent, otherwise add its negation.
\end{proof}

\section{The Canonical Model}

\begin{definition}[Canonical Model]
The \emph{canonical model} for logic $L$ is $\mathcal{M}^c_L = (W^c, R^c, V^c)$ where:
\begin{itemize}
    \item $W^c$ = all maximally $L$-consistent sets
    \item $\Gamma R^c \Delta$ iff for all $\varphi$: $\necessary\varphi \in \Gamma \Rightarrow \varphi \in \Delta$
    \item $V^c(p) = \{\Gamma : p \in \Gamma\}$
\end{itemize}
\end{definition}

\begin{lemma}[Truth Lemma]
$\mathcal{M}^c_L, \Gamma \models \varphi$ iff $\varphi \in \Gamma$.
\end{lemma}

\begin{proof}
Induction on $\varphi$. The key case is $\necessary\varphi$:
\begin{itemize}
    \item If $\necessary\varphi \in \Gamma$ and $\Gamma R^c \Delta$, then $\varphi \in \Delta$, so by IH $\Delta \models \varphi$.
    \item If $\necessary\varphi \notin \Gamma$, construct $\Delta$ with $\varphi \notin \Delta$ and $\Gamma R^c \Delta$. % TODO: details
\end{itemize}
\end{proof}

\section{Completeness Theorem}

\begin{theorem}[Completeness of K]
If $\varphi$ is valid in all Kripke frames, then $\varphi$ is provable in K.
\end{theorem}

\begin{proof}
Contrapositive. If $\varphi$ is not provable, then $\{\neg\varphi\}$ is consistent. Extend to MCS $\Gamma$. By Truth Lemma, $\mathcal{M}^c_K, \Gamma \models \neg\varphi$, so $\varphi$ is not valid.
\end{proof}

\section{Canonicity}

\begin{definition}[Canonical Formula]
A formula $\varphi$ is \emph{canonical} if: whenever $\varphi \in L$, the canonical frame for $L$ validates $\varphi$.
\end{definition}

\begin{keyinsight}
Canonicity is the bridge between syntax (axioms) and semantics (frame conditions). If an axiom is canonical, adding it to a logic preserves completeness.
\end{keyinsight}

\begin{example}
The T axiom $\necessary p \to p$ is canonical: if T $\in L$, then $R^c$ is reflexive.
\end{example}

% TODO: more on canonicity, Sahlqvist
