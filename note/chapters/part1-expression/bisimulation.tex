\chapter{Bisimulation}

\begin{goals}
\begin{itemize}
    \item Understand bisimulation as the fundamental equivalence for modal logic
    \item Master the zig-zag conditions
    \item See bisimulation games as an intuitive characterization
    \item Connect to the later coalgebraic generalization
\end{itemize}
\end{goals}

\section{The Problem of Equivalence}

When are two models ``the same'' from a modal perspective?

Not isomorphism---too strict. Two very different-looking models might satisfy exactly the same formulas.

\begin{example}
Consider:
\begin{center}
\begin{tikzpicture}[world/.style={circle, draw, minimum size=6mm}, arrow/.style={-{Stealth}}]
\node[world] (w) at (0,0) {$w$};
\node[world] (v1) at (1.5,0.5) {$v_1$};
\node[world] (v2) at (1.5,-0.5) {$v_2$};
\draw[arrow] (w) -- (v1);
\draw[arrow] (w) -- (v2);

\node[world] (w') at (4,0) {$w'$};
\node[world] (v') at (5.5,0) {$v'$};
\draw[arrow] (w') -- (v');
\end{tikzpicture}
\end{center}
If $v_1, v_2, v'$ all satisfy the same atoms, then $w$ and $w'$ satisfy the same modal formulas---even though the models have different structure.
\end{example}

\section{Definition}

\begin{definition}[Bisimulation]
Let $\mathcal{M} = (W, R, V)$ and $\mathcal{M}' = (W', R', V')$. A relation $Z \subseteq W \times W'$ is a \emph{bisimulation} if whenever $w \mathrel{Z} w'$:

\textbf{(Atoms)} For all $p \in \Phi$: $w \in V(p) \Leftrightarrow w' \in V'(p)$

\textbf{(Zig)} If $wRv$, then there exists $v' \in W'$ such that $w'R'v'$ and $v \mathrel{Z} v'$

\textbf{(Zag)} If $w'R'v'$, then there exists $v \in W$ such that $wRv$ and $v \mathrel{Z} v'$
\end{definition}

We write $w \bisim w'$ if there exists a bisimulation relating $w$ and $w'$.

\begin{intuition}
Zig: ``I can match your move.'' \\
Zag: ``You can match my move.'' \\
Two states are bisimilar if they can ``simulate'' each other, step by step.
\end{intuition}

\section{Bisimulation Games}

Bisimulation has a game-theoretic interpretation.

\textbf{Players}: Spoiler (tries to distinguish) vs Duplicator (tries to match)

\textbf{Game}: Start at $(w, w')$. Each round:
\begin{enumerate}
    \item Spoiler picks a side and makes a move (follows an edge)
    \item Duplicator must match on the other side
\end{enumerate}

\textbf{Winning}:
\begin{itemize}
    \item Spoiler wins if atoms differ, or Duplicator can't move
    \item Duplicator wins if the game continues forever
\end{itemize}

\begin{theorem}
$w \bisim w'$ iff Duplicator has a winning strategy.
\end{theorem}

\section{The Invariance Theorem}

\begin{theorem}[Bisimulation Invariance]
If $w \bisim w'$, then for all modal formulas $\varphi$:
\[
\mathcal{M}, w \models \varphi \iff \mathcal{M}', w' \models \varphi
\]
\end{theorem}

\begin{proof}
Induction on $\varphi$.

\textbf{Base}: $\varphi = p$. By (Atoms).

\textbf{Step $\neg$}: Immediate from IH.

\textbf{Step $\land$}: Immediate from IH.

\textbf{Step $\necessary$}: Suppose $\mathcal{M}, w \models \necessary\psi$ and $w \mathrel{Z} w'$.
\begin{itemize}
    \item Take any $v'$ with $w'R'v'$
    \item By (Zag), exists $v$ with $wRv$ and $v \mathrel{Z} v'$
    \item Since $\mathcal{M}, w \models \necessary\psi$, we have $\mathcal{M}, v \models \psi$
    \item By IH, $\mathcal{M}', v' \models \psi$
\end{itemize}
So $\mathcal{M}', w' \models \necessary\psi$. The other direction uses (Zig).
\end{proof}

\section{Largest Bisimulation}

\begin{proposition}
The union of all bisimulations is itself a bisimulation---the \emph{largest} bisimulation.
\end{proposition}

\begin{definition}
$w \sim w'$ (bisimilarity) iff $(w, w')$ is in the largest bisimulation.
\end{definition}

\section{Looking Ahead: Coalgebraic Bisimulation}

The zig-zag conditions generalize beyond Kripke frames.

For any coalgebra $(X, \gamma : X \to FX)$, there's a notion of bisimulation. For the powerset functor $F = \mathcal{P}$, it coincides exactly with Kripke bisimulation.

\begin{keyinsight}
Bisimulation is not an accident of Kripke semantics. It's a fundamental concept that exists for \emph{any} coalgebra. This is why it will reappear throughout the book.
\end{keyinsight}
