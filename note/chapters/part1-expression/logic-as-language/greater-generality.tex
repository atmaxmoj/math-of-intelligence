% Section: The Need for Greater Generality

Remember the thread: the desire for greater expressive power.

Let us take stock.

Logic has transformed from the study of valid argument into a family of languages for describing structured systems:
\begin{itemize}
    \item Modal logic describes systems of possible worlds
    \item Temporal logic describes systems evolving in time
    \item Dynamic logic describes systems of program states
    \item Epistemic logic describes systems of knowledge states
\end{itemize}

Each logic is tailored to a particular kind of structure.

But this raises a question: \textbf{is there a language that can describe structure itself?}

Not this or that particular structure, but the general notion of ``what it means to have structure'' and ``what it means to preserve structure.''

\begin{intuition}
We have many lenses, each suited to viewing a particular kind of system.

Can we build a \textbf{lens for lenses}---a framework for talking about what all these lenses have in common?
\end{intuition}

The four pillars of logic (proof theory, model theory, computability, set theory) all use set theory as their metalanguage. But set theory talks about \emph{membership}---what elements belong to what collections. This is not the same as \emph{structure}.

What we need is a language where \textbf{structure} and \textbf{invariance under transformation} are the primitive concepts.
