% Section: Why This Matters for Us

This book is about \textbf{the algebra of intelligence}. We want to understand:
\begin{itemize}
    \item How can agents reason about possibility, knowledge, change?
    \item How can we specify and verify agent behavior?
    \item How can formal structures \emph{learn}?
\end{itemize}

Modal logic gives us the \textbf{language}.

Coalgebra gives us the \textbf{semantics}---a unified framework for all kinds of state-based, behavioral systems.

And the algebraic/coalgebraic duality will be crucial when we ask: how do we make these structures \textbf{learnable}?

But first, we need to understand modal logic properly. That begins in the next chapter, with Kripke frames.

\begin{summary}
The history of logic is the history of the desire for greater expressive power.

\begin{itemize}
    \item \textbf{The ladder}: syllogisms $\to$ propositional $\to$ first-order $\to$ modal $\to$ ...
    \item \textbf{The shift}: from ``studying consequence'' to ``describing structured systems''
    \item \textbf{The convergence}: philosophers, linguists, computer scientists all need to describe systems
    \item \textbf{The ultimate abstraction}: category theory---the language of structure itself
    \item \textbf{The duality}: algebra (syntax, construction) vs. coalgebra (behavior, observation)
    \item \textbf{The connection}: modal logic is the natural language for coalgebraic systems
\end{itemize}

One thread runs through it all: \emph{I can't say what I want to say---so I build a more expressive language.}
\end{summary}
