% Section: Modal Logic as Coalgebraic Logic

Here is where everything connects.

Remember Kripke semantics for modal logic? A \textbf{Kripke frame} is a pair $(W, R)$ where:
\begin{itemize}
    \item $W$ is a set of ``possible worlds''
    \item $R \subseteq W \times W$ is an ``accessibility relation''
\end{itemize}

But this is exactly a \textbf{coalgebra}! A Kripke frame is a coalgebra for the powerset functor:
\[
\alpha: W \to \mathcal{P}(W)
\]
where $\alpha(w) = \{v \mid w R v\}$---the set of worlds accessible from $w$.

\begin{keyinsight}
\textbf{Kripke frames are coalgebras.}

And the key notion of equivalence for coalgebras is \textbf{bisimulation}---which is exactly the right notion of equivalence for modal logic!

Van Benthem's theorem: modal logic is the \textbf{bisimulation-invariant} fragment of first-order logic.
\end{keyinsight}

This is not a coincidence. Modal logic is the \textbf{natural language for describing coalgebraic behavior}.

Different modal logics arise from different functors:
\begin{itemize}
    \item Kripke frames: $F(X) = \mathcal{P}(X)$
    \item Labeled transition systems: $F(X) = \mathcal{P}(A \times X)$ for action set $A$
    \item Probabilistic systems: $F(X) = \mathcal{D}(X)$ (probability distributions)
    \item Automata: $F(X) = 2 \times X^A$ (accepting/rejecting + transitions)
\end{itemize}

\textbf{Coalgebraic modal logic} provides a uniform treatment of all these cases.
