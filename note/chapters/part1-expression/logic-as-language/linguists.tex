% Section: Enter the Linguists

Meanwhile, in linguistics, a parallel development was happening.

\subsection{Chomsky: Language Has Hidden Structure}

Noam Chomsky revolutionized linguistics in the 1950s with a simple observation: \textbf{language is not a list of sentences}. It is a \textbf{generative system}---a finite set of rules that produces infinitely many sentences.

\begin{history}
Chomsky's \emph{Syntactic Structures} (1957) introduced formal grammars to linguistics. The same mathematical tools used for programming languages could describe natural languages.
\end{history}

But Chomsky focused on \textbf{syntax}---the structure of sentences. What about \textbf{meaning}?

\subsection{Montague: Natural Language Has Formal Semantics}

Richard Montague made a bold claim: natural language can be given a \textbf{precise formal semantics}, just like a programming language.

\begin{history}
``I reject the contention that an important theoretical difference exists between formal and natural languages.'' --- Richard Montague, 1970
\end{history}

Montague grammar uses \textbf{intensional logic}---a form of modal logic---to analyze meaning. Why modal logic?

Because natural language is full of \textbf{modality}:
\begin{itemize}
    \item ``John \emph{might} come'' (possibility)
    \item ``Mary \emph{must} have left'' (necessity/inference)
    \item ``He \emph{believes} that it's raining'' (propositional attitudes)
    \item ``She \emph{will} arrive tomorrow'' (future tense)
\end{itemize}

\begin{keyinsight}
Natural language doesn't just describe what \emph{is}. It describes what \emph{could be}, what \emph{must be}, what someone \emph{thinks} is, what \emph{will be}.

Modal logic is the natural tool for this, because it has the expressive power to talk about \textbf{alternative possibilities}.
\end{keyinsight}
