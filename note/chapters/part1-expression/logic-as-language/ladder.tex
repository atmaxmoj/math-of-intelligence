% Section: The Ladder of Expressiveness

Logic began with a simple question: which arguments are valid? But the answer required building formal \textbf{languages}---and over time, these languages became more and more expressive.

Each step up the ladder was driven by the same complaint: \textbf{I can't say what I want to say.}

This desire for greater expressive power is the thread that runs through the entire history of logic---from Aristotle to category theory. Keep it in mind as we climb.

\subsection{Syllogisms: The Starting Point}

Aristotle's syllogisms could express statements of the form:
\begin{itemize}
    \item All A are B
    \item Some A are B
    \item No A are B
    \item Some A are not B
\end{itemize}

This is enough for many philosophical arguments. But try to express:
\begin{center}
``Every person loves someone.''
\end{center}

Is it ``All persons are lovers''? That loses the structure. The syllogism sees only \textbf{two terms} in a fixed relationship. The internal complexity---that loving involves a \emph{relation} between two things---is invisible.

\subsection{Propositional Logic: From the Stoics to Boole}

\begin{history}
Propositional logic actually predates Boole by two millennia. The \textbf{Stoic} philosophers (3rd century BC) studied arguments based on connectives like ``and,'' ``or,'' and ``if-then''---independently of Aristotle's term logic.

They identified basic argument forms, including what we now call \emph{modus ponens} and \emph{modus tollens}. But their work was largely forgotten, rediscovered only in the 20th century.

George Boole (1854) reinvented propositional logic algebraically, treating propositions as variables and connectives as operations. This algebraic approach made logic part of mathematics.
\end{history}

Boole's algebra could express combinations of propositions:
\[
(p \land q) \to r
\]

The basic \textbf{inference rules} of propositional logic are:

\begin{description}
    \item[Modus Ponens] From $P$ and $P \to Q$, infer $Q$.\\
    \textit{``It rains. If it rains, the ground is wet. So the ground is wet.''}

    \item[Modus Tollens] From $\neg Q$ and $P \to Q$, infer $\neg P$.\\
    \textit{``The ground is dry. If it rains, the ground is wet. So it's not raining.''}

    \item[Hypothetical Syllogism] From $P \to Q$ and $Q \to R$, infer $P \to R$.\\
    \textit{``If A then B. If B then C. So if A then C.'' (Chaining conditionals)}

    \item[Disjunctive Syllogism] From $P \lor Q$ and $\neg P$, infer $Q$.\\
    \textit{``A or B. Not A. So B.''}
\end{description}

These rules are \textbf{truth-preserving}: if the premises are true, the conclusion must be true. This can be verified by \textbf{truth tables}. Each connective is defined by how the truth value of the whole depends on the truth values of the parts:

\begin{center}
\begin{tabular}{cc|c}
$p$ & $q$ & $p \to q$ \\
\hline
T & T & T \\
T & F & F \\
F & T & T \\
F & F & T
\end{tabular}
\end{center}

Wait---why is $p \to q$ true when $p$ is false?

\begin{intuition}
This is \textbf{material implication}, and it confuses everyone at first.

Think of implication as a \textbf{rule}: ``When the light is red, all cars must stop.''

Now suppose the traffic light is broken today---it's green all day. No car stops. Has anyone violated the rule?

\textbf{No.} The rule only says what must happen \emph{when the light is red}. If the light is never red, the rule is never triggered. It's not violated---it simply doesn't apply.

That's material implication: $p \to q$ is false \emph{only} when $p$ is true and $q$ is false. Otherwise, no violation.
\end{intuition}

This leads to strange truths. Let $p$ = ``the moon is made of cheese.'' Then:
\[
p \to q
\]
is \textbf{true} for \emph{any} $q$---``if the moon is made of cheese, then I am the Pope,'' ``if the moon is made of cheese, then $2+2=5$.'' All true, because $p$ is false.

In fact, from a false premise, you can ``derive'' anything. This is called \term{ex falso quodlibet} (from falsehood, anything follows).

Even worse, you can construct seemingly valid ``proofs'' of absurd conclusions---like proving that God exists from the mere fact that you don't pray. We'll analyze this famous example in detail after introducing modal logic, where we'll see how a more expressive language dissolves the paradox.

But for now, note the more basic limitation: each proposition is atomic. ``All men are mortal'' is just a letter $p$. You cannot look inside to see the quantifier ``all'' or the predicate ``mortal.''

\subsection{First-Order Logic: Frege's Revolution}

Frege's first-order logic finally cracked open propositions:
\[
\forall x \, (\mathrm{Man}(x) \to \mathrm{Mortal}(x))
\]

Now we have:
\begin{itemize}
    \item \textbf{Variables}: $x, y, z$ ranging over individuals
    \item \textbf{Predicates}: $\mathrm{Man}(x)$, $\mathrm{Mortal}(x)$
    \item \textbf{Relations}: $\mathrm{Loves}(x, y)$
    \item \textbf{Quantifiers}: $\forall$ (for all), $\exists$ (exists)
\end{itemize}

``Every person loves someone'' becomes:
\[
\forall x \, (\mathrm{Person}(x) \to \exists y \, \mathrm{Loves}(x, y))
\]

First-order logic is \textbf{remarkably expressive}. You can formalize most of mathematics in it. For decades, it was \emph{the} logic.

But it inherits the same quirk from propositional logic: vacuous truth. Consider:
\begin{quote}
``All even primes other than 2 are divisible by 5.''
\end{quote}
Formally: $\forall x \, ((\mathrm{EvenPrime}(x) \land x \neq 2) \to \mathrm{DivisibleBy5}(x))$.

This is \textbf{true}. Why? Because there are no even primes other than 2. The antecedent is never satisfied, so the implication is vacuously true for all $x$.

Or consider: ``I am taller than every blue giraffe in this room.'' Formally:
\[
\forall x \, ((\mathrm{BlueGiraffe}(x) \land \mathrm{InThisRoom}(x)) \to \mathrm{TallerThan}(me, x))
\]
Also true---there are no blue giraffes here. You can truthfully claim to be taller than all of them.

Why does this work? Because of the \textbf{algebraic structure} of quantifiers. The negation of our claim would be:
\[
\exists x \, ((\mathrm{BlueGiraffe}(x) \land \mathrm{InThisRoom}(x)) \land \neg\mathrm{TallerThan}(me, x))
\]
In plain English: ``There exists a blue giraffe in this room that is at least as tall as me.''

This is clearly \textbf{false}---there are no blue giraffes at all. And since the negation is false, the original statement must be true.

\begin{intuition}
This feels strange because our intuition says: ``How can you be taller than something that doesn't exist?'' But the logic doesn't say you \emph{are} taller than anything---it says there's \emph{no counterexample}. And indeed there isn't: no blue giraffe stands in this room being taller than you.

The algebraic duality $\neg\forall x\, \varphi \equiv \exists x\, \neg\varphi$ forces this. To deny ``all X have property P,'' you must produce an X that lacks P. If no X exists, you can't produce one, so the universal claim stands.
\end{intuition}

In mathematics, vacuous truth is often convenient (it lets us state theorems without worrying about edge cases). But it shows that first-order logic's $\to$ still doesn't match our intuitive ``if-then.''

And there's more it cannot say.

\subsection{What First-Order Logic Cannot Say}

Consider these statements:
\begin{itemize}
    \item ``It is \emph{necessary} that $2 + 2 = 4$.''
    \item ``Alice \emph{knows} that Bob is lying.''
    \item ``The program \emph{will eventually} terminate.''
    \item ``It is \emph{obligatory} to keep promises.''
\end{itemize}

These all involve a \textbf{mode} of truth. Not just ``is $\varphi$ true?'' but ``in what \emph{way} is $\varphi$ true?''

First-order logic has no way to express these modes. It knows only bare, unqualified truth.

\subsection{Modal Logic: Beyond First-Order}

Modal logic adds \textbf{operators} that modify propositions:
\begin{itemize}
    \item $\Box \varphi$: necessarily $\varphi$ / in all accessible worlds, $\varphi$
    \item $\Diamond \varphi$: possibly $\varphi$ / in some accessible world, $\varphi$
\end{itemize}

With different interpretations of ``accessible,'' we get different modal logics:

\begin{center}
\begin{tabular}{lll}
\textbf{Interpretation} & $\Box\varphi$ means & \textbf{Field} \\
\hline
Necessity & necessarily $\varphi$ & Metaphysics \\
Knowledge & agent knows $\varphi$ & Epistemology \\
Time & always in the future, $\varphi$ & Temporal logic \\
Obligation & it ought to be that $\varphi$ & Deontic logic \\
After action $a$ & after doing $a$, $\varphi$ holds & Dynamic logic \\
\end{tabular}
\end{center}

\begin{intuition}
Modal logic is not one logic but a \textbf{family} of logics, each tuned to a different notion of ``possibility'' and ``necessity.'' The syntax ($\Box$, $\Diamond$) is the same; the semantics varies.
\end{intuition}

This explosion of modal logics---hundreds of them, with different axioms and different applications---raises a question:

\textbf{Is there a unified framework for all of them?}

But first, let us see how modal logic resolves the paradox we encountered earlier.

\subsection{Revisiting the Prayer Argument}

Recall the ``proof'' that God exists. Here is the full derivation, with both the formal logic and plain English at each step:

\begin{enumerate}
    \item $\neg G \to \neg (P \to A)$ \hfill (Premise 1)

    \textit{``If God doesn't exist, then `if I pray, I'll be answered' is false.''}

    \item $\neg P$ \hfill (Premise 2)

    \textit{``I don't pray.''}

    \item $P \to A$ \hfill (From 2, by truth table of $\to$)

    \textit{``Since I don't pray, `if I pray, I'll be answered' is vacuously true.'' (The antecedent is false, so the conditional is true.)}

    \item $(P \to A) \to G$ \hfill (Contrapositive of 1)

    \textit{``If `if I pray, I'll be answered' is true, then God exists.'' (This is logically equivalent to Premise 1.)}

    \item $G$ \hfill (Modus ponens on 3 and 4)

    \textit{``Therefore, God exists.''}
\end{enumerate}

The logic is valid. The conclusion is absurd. What went wrong?

\subsection{The Modal Solution: Strict Implication}

The problem is that material implication $P \to A$ is too coarse to express the nuance we actually need. It only talks about the \emph{actual} world. In the actual world, I don't pray, so $P \to A$ is trivially true.

But when we say ``if I pray, I'll be answered,'' we mean something stronger: \emph{in any situation where I pray, I would be answered}. This is \textbf{strict implication}:
\[
\Box(P \to A)
\]

Let's be explicit about the notation:
\begin{itemize}
    \item $\Box$ reads as ``necessarily'' or ``in all possible situations''
    \item $\Box(P \to A)$ reads as: ``Necessarily, if I pray then I'm answered''
    \item In plain English: ``In \emph{every} possible situation where I pray, I would be answered''
\end{itemize}

This is much stronger than material implication $P \to A$, which only talks about the actual world.

Now let's redo the argument with strict implication, step by step:

\textbf{Premise 1:} ``If God doesn't exist, then it's \emph{not} necessarily true that praying leads to being answered.''
\[ \neg G \to \neg \Box(P \to A) \]

\textbf{Premise 2:} ``I don't pray (in the actual world).''
\[ \neg P \]

Now can we derive $G$ (God exists)?

\textbf{Step 1:} ``Since I don't pray, `if I pray then I'm answered' is trivially true---here, now.''
\[ \neg P \;\vdash\; P \to A \quad \text{(in the actual world)} \]

\textbf{Step 2:} ``But does `I don't pray here' imply `in ALL situations, praying leads to answers'?''
\[ \neg P \;\vdash\; \Box(P \to A) \quad \textbf{???} \]

\textbf{No!} This does not follow. In other possible situations, I might pray.

\textbf{Step 3:} Without $\Box(P \to A)$, we cannot use the contrapositive of Premise 1. The trick that ``proved'' God exists no longer works.

\textbf{Why the argument collapses:} In other possible situations, I might pray. And in those situations, if God doesn't exist, my prayers wouldn't be answered. So $P \to A$ would be \emph{false} there---even though it's vacuously true here.

Therefore: $\neg P \nvdash \Box(P \to A)$. ``I don't pray'' does \emph{not} imply ``necessarily, praying leads to answers.''

Modal logic has the expressive power to distinguish what classical logic conflates.

\subsection{The Moral: Expressiveness Drives Progress}

\begin{history}
This is exactly why C.I. Lewis invented modal logic in 1912.\footnote{See \href{https://plato.stanford.edu/entries/logic-modal-origins/}{Stanford Encyclopedia of Philosophy: Modern Origins of Modal Logic}.} He was disturbed by the ``paradoxes of material implication'' in Russell and Whitehead's \emph{Principia Mathematica}: that a false proposition implies anything, and a true proposition is implied by anything.

Lewis argued that this doesn't match the meaning of ``implies'' in ordinary reasoning and proof. He developed \textbf{strict implication} $A \Rightarrow B \;=_{\text{def}}\; \Box(A \to B)$ to capture a stronger notion. Kripke later provided the elegant possible-worlds semantics that made modal logic mathematically tractable.
\end{history}

\begin{keyinsight}
Here is the crucial point:

We did \emph{not} say: ``The prayer argument is logically valid in propositional logic, so we must accept that God exists.''

We said: ``The prayer argument reveals that propositional logic's notion of validity doesn't match our intuitions about implication. So we build a better logic.''

\textbf{Logic is a tool, not a master.} When the tool doesn't fit the job---when its $\vdash$ doesn't match our intuitive sense of ``follows from''---we develop a more expressive language. This is the engine that drives the history of logic.
\end{keyinsight}
