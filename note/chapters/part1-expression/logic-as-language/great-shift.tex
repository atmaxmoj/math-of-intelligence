% Section: The Great Shift: From Consequence to Structure

Before we continue, let us pause to notice something profound.

Look at how the question has changed:

\begin{center}
\begin{tabular}{ll}
\textbf{Aristotle's question:} & Is this argument valid? \\
& (Does the conclusion follow from the premises?) \\[1em]
\textbf{The modern question:} & What kind of structure does this logic describe? \\
& (What are the ``worlds'' and how are they related?)
\end{tabular}
\end{center}

This is a fundamental shift in what logic \emph{is}.

\subsection{The Old View: Logic Studies Consequence}

In the old view, logic was about \textbf{valid inference}:
\begin{quote}
Given premises $A_1, \ldots, A_n$, does conclusion $B$ necessarily follow?
\end{quote}

The central concept was \textbf{consequence}: the relation $A_1, \ldots, A_n \vdash B$.

Everything else---syntax, semantics, proof systems---served this goal.

\subsection{The New View: Logic Describes Systems}

In the new view, logic is a \textbf{language for describing structured systems}:
\begin{quote}
Given a type of system (possible worlds, time, knowledge states, program states...), how do we talk about it precisely?
\end{quote}

The central concept is \textbf{structure}: what kind of ``world'' does this logic describe?

Consequence becomes \textbf{derived}: once you fix a logic and its class of structures, the consequence relation falls out automatically.

\begin{keyinsight}
Logic shifted from studying \textbf{``what follows from what''} to studying \textbf{``how to describe certain kinds of systems.''}

Consequence is no longer the definition---it is a \emph{consequence} of the definition.
\end{keyinsight}

\subsection{When Did This Happen?}

The shift was gradual, but modal logic made it explicit.

In classical propositional or first-order logic, you can almost ignore the ``worlds.'' There's just one world, and formulas are either true or false in it. The focus stays on consequence.

But modal logic \emph{forces} you to think about structure:
\begin{itemize}
    \item What are the possible worlds?
    \item What is the accessibility relation?
    \item Is it reflexive? Transitive? Symmetric?
\end{itemize}

The logic doesn't make sense until you specify the \textbf{structure}. And different structures give different logics.

\begin{intuition}
Think of it this way:

\textbf{Old:} Logic is a machine for checking arguments.

\textbf{New:} Logic is a lens for viewing systems. Different logics are different lenses, suited to different systems.

When you ``apply'' a logic to a domain, you are \emph{instantiating} the lens---choosing a particular system to view through it. Consequence is what you see through the lens.
\end{intuition}

\subsection{Why This Matters}

This shift explains why so many different fields need logic.

They don't need logic because they want to ``do deduction.'' They need logic because they have \textbf{structured systems} to describe:

\begin{itemize}
    \item \textbf{Linguists}: Natural language is a system. Sentences have structure. ``If this word is a verb, what can come next?'' is a structural question.
    \item \textbf{Computer scientists}: Programs are systems. States, transitions, specifications---all structural.
    \item \textbf{Philosophers}: Possible worlds, knowledge states, moral situations---structured domains that need precise description.
\end{itemize}

Logic provides the \textbf{general framework} for describing systems with antecedent-consequent structure: ``in situation X, Y holds'' or ``from state X, you can reach state Y.''

This is not ``reasoning'' in the everyday sense. It is \textbf{structural description}.
