\chapter{Correspondence and Sahlqvist's Theorem}

\begin{goals}
\begin{itemize}
    \item Understand frame correspondence deeply
    \item Learn what Sahlqvist formulas are
    \item Appreciate why Sahlqvist's theorem is powerful
    \item Know that this has been mechanized
\end{itemize}
\end{goals}

\section{The Correspondence Problem}

Recall: some modal axioms correspond to first-order frame conditions.

\begin{center}
\begin{tabular}{ll}
$\necessary p \to p$ & $\forall x. Rxx$ \\
$\necessary p \to \necessary\necessary p$ & $\forall xyz. Rxy \land Ryz \to Rxz$ \\
$\possible p \to \necessary\possible p$ & $\forall xy. Rxy \to \forall z. Rxz \to Ryz$ \\
\end{tabular}
\end{center}

\textbf{Question}: Which modal formulas correspond to first-order conditions?

\textbf{Answer}: Not all! The L\"ob axiom $\necessary(\necessary p \to p) \to \necessary p$ corresponds to a \emph{second-order} condition (well-foundedness).

\section{Sahlqvist Formulas}

\begin{definition}[Sahlqvist Formula (simplified)]
A \emph{Sahlqvist formula} has the form:
\[
\varphi \to \psi
\]
where:
\begin{itemize}
    \item $\varphi$ is \emph{positive} in $\necessary$ and built from boxed atoms ($\necessary^n p$), negative formulas, and $\land, \lor$
    \item $\psi$ is \emph{positive} (no negations in front of atoms)
\end{itemize}
\end{definition}

\begin{example}
All the standard axioms are Sahlqvist:
\begin{itemize}
    \item $\necessary p \to p$ (T)
    \item $\necessary p \to \necessary\necessary p$ (4)
    \item $p \to \necessary\possible p$ (B)
    \item $\possible p \to \necessary\possible p$ (5)
\end{itemize}
\end{example}

\section{Sahlqvist's Theorem}

\begin{theorem}[Sahlqvist 1975]
Every Sahlqvist formula:
\begin{enumerate}
    \item Corresponds to a first-order frame condition (computable from the formula)
    \item Is canonical (hence its logic is complete)
\end{enumerate}
\end{theorem}

\begin{keyinsight}
Sahlqvist's theorem is a ``cookbook'' for modal logic:
\begin{enumerate}
    \item Write down your axiom (if Sahlqvist)
    \item Automatically get the frame condition
    \item Automatically get completeness
\end{enumerate}
No need to construct canonical models by hand!
\end{keyinsight}

\section{The Algorithm (SQEMA)}

There's an algorithm called \textbf{SQEMA} (Sahlqvist-van Benthem algorithm) that:
\begin{itemize}
    \item Takes a modal formula as input
    \item Outputs the corresponding first-order condition (if it exists)
\end{itemize}

The algorithm works by:
\begin{enumerate}
    \item Translate to first-order logic with the standard translation
    \item Apply Ackermann's lemma to eliminate second-order quantifiers
    \item If successful, output the first-order result
\end{enumerate}

\section{Mechanization}

\begin{history}
Sahlqvist's theorem has been formalized in proof assistants:
\begin{itemize}
    \item Isabelle/HOL: formalized by various authors
    \item Lean: partial formalizations exist
\end{itemize}
This is one of the benchmark results for modal logic mechanization.
\end{history}

\begin{exercise}
\begin{enumerate}
    \item Verify that the McKinsey axiom $\necessary\possible p \to \possible\necessary p$ is \emph{not} Sahlqvist. (Hint: it's not canonical for the expected frame class.)
    \item Use the standard translation to find the frame condition for $\necessary(p \lor q) \to \necessary p \lor \necessary q$. Is this formula valid on all frames?
\end{enumerate}
\end{exercise}

% TODO: more on SQEMA, examples of the algorithm
