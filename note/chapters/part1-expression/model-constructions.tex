\chapter{Model Constructions}

\begin{goals}
\begin{itemize}
    \item Master bisimulation and its equivalent characterizations
    \item Understand bounded morphisms (p-morphisms)
    \item Learn model constructions: generated submodels, disjoint unions, unravelling
    \item Prove the Hennessy-Milner theorem
\end{itemize}
\end{goals}

\section{Bisimulation Revisited}

Recall: a bisimulation relates states that are ``behaviorally equivalent.''

\begin{definition}[Bisimulation]
$Z \subseteq W \times W'$ is a bisimulation if $wZw'$ implies:
\begin{enumerate}
    \item \textbf{(Atoms)} Same atomic propositions
    \item \textbf{(Zig)} Every successor of $w$ is matched by a successor of $w'$
    \item \textbf{(Zag)} Every successor of $w'$ is matched by a successor of $w$
\end{enumerate}
\end{definition}

\begin{theorem}[Bisimulation Invariance]
Bisimilar states satisfy the same modal formulas.
\end{theorem}

\begin{proof}
Induction on formula structure.
% TODO: full proof
\end{proof}

\section{Bounded Morphisms}

\begin{definition}[Bounded Morphism / p-morphism]
A function $f : W \to W'$ is a \emph{bounded morphism} if:
\begin{enumerate}
    \item $w \in V(p) \Leftrightarrow f(w) \in V'(p)$
    \item $wRv \Rightarrow f(w)R'f(v)$ \hfill (homomorphism)
    \item $f(w)R'v' \Rightarrow \exists v: wRv \land f(v) = v'$ \hfill (back condition)
\end{enumerate}
\end{definition}

\begin{proposition}
If $f$ is a bounded morphism, then its graph is a bisimulation.
\end{proposition}

\section{Generated Submodels}

\begin{definition}[Generated Submodel]
The submodel \emph{generated by} $w$ is the restriction to all worlds reachable from $w$.
\end{definition}

\begin{theorem}
A world satisfies the same formulas in the full model and in its generated submodel.
\end{theorem}

\section{Disjoint Union}

\begin{definition}
The \emph{disjoint union} $\mathcal{M}_1 \uplus \mathcal{M}_2$ combines two models without connecting them.
\end{definition}

\begin{proposition}
Each component of a disjoint union is bisimilar to itself in the union.
\end{proposition}

\section{Tree Unravelling}

\begin{definition}[Unravelling]
The \emph{unravelling} of $\mathcal{M}$ from $w$ is a tree where:
\begin{itemize}
    \item Nodes are finite paths starting from $w$
    \item Edges extend paths by one step
    \item Valuation inherited from endpoints
\end{itemize}
\end{definition}

\begin{theorem}
Every pointed model is bisimilar to its unravelling.
\end{theorem}

\begin{keyinsight}
Modal logic cannot distinguish a model from its tree unravelling. This is why modal logic has the \emph{tree model property}.
\end{keyinsight}

\section{Hennessy-Milner Theorem}

\begin{theorem}[Hennessy-Milner]
On \emph{image-finite} models, bisimilarity coincides with modal equivalence.
\end{theorem}

\begin{intuition}
On finite-branching models, if two states satisfy exactly the same modal formulas, they must be bisimilar. The converse always holds; this theorem says that for nice enough models, the two notions coincide exactly.
\end{intuition}

% TODO: proof, counterexample for non-image-finite
