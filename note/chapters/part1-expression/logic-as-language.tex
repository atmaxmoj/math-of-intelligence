\chapter{Logic as Language}

\begin{goals}
\begin{itemize}
    \item Understand logic as a formal language for making statements
    \item See why propositional logic is limited
    \item Motivate the need for modal operators
\end{itemize}
\end{goals}

\section{What Can We Say?}

Logic is a language. Like any language, it has:
\begin{itemize}
    \item \textbf{Syntax}: the grammar---what counts as a well-formed sentence
    \item \textbf{Semantics}: the meaning---what sentences are true and when
\end{itemize}

\begin{intuition}
Think of logic as a very precise, very restricted language. It can't express poetry or emotion, but what it \emph{can} express, it expresses unambiguously. This precision is its power.
\end{intuition}

\section{Propositional Logic}

The simplest logic. We have propositions ($p, q, r, \ldots$) and connectives ($\neg, \land, \lor, \to$).

\begin{definition}[Propositional Language]
Let $\Phi$ be a countable set of propositional variables. The propositional language $\mathcal{L}_0$ is:
\[
\varphi ::= p \mid \neg\varphi \mid (\varphi \land \psi)
\]
where $p \in \Phi$. Other connectives are defined: $\varphi \lor \psi := \neg(\neg\varphi \land \neg\psi)$, etc.
\end{definition}

Semantics: a \emph{valuation} $v : \Phi \to \{0, 1\}$ assigns truth values.

\begin{warning}
Propositional logic is \emph{flat}. Every proposition is either true or false, period. There's no room for ``true from one perspective but false from another.'' Real reasoning needs more.
\end{warning}

\section{The Need for Modality}

Consider these statements:
\begin{itemize}
    \item ``It is \emph{possible} that it will rain tomorrow.''
    \item ``Alice \emph{knows} that Bob is lying.''
    \item ``It is \emph{obligatory} to pay taxes.''
    \item ``The program \emph{will eventually} terminate.''
\end{itemize}

None of these can be expressed in propositional logic. They all involve a \emph{mode} of truth.

\begin{keyinsight}
Modal logic adds \textbf{operators} that modify propositions:
\begin{itemize}
    \item $\necessary\varphi$: ``necessarily $\varphi$'' / ``in all accessible states, $\varphi$''
    \item $\possible\varphi$: ``possibly $\varphi$'' / ``in some accessible state, $\varphi$''
\end{itemize}
\end{keyinsight}

\section{Modal Language}

\begin{definition}[Modal Language]
The modal language $\mathcal{L}(\necessary, \possible)$ extends propositional logic:
\[
\varphi ::= p \mid \neg\varphi \mid (\varphi \land \psi) \mid \necessary\varphi
\]
We define $\possible\varphi := \neg\necessary\neg\varphi$.
\end{definition}

But what does $\necessary\varphi$ \emph{mean}? For that, we need \textbf{possible worlds}.
