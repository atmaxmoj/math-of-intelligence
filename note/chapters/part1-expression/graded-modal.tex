\chapter{Graded Modal Logic}

\begin{goals}
\begin{itemize}
    \item Go beyond $\forall$/$\exists$: counting quantifiers
    \item Understand graded modalities: $\possible_{\geq n}$, $\possible_{\leq n}$
    \item See majority and probabilistic quantifiers
    \item Appreciate the trade-offs (expressiveness vs decidability)
\end{itemize}
\end{goals}

\section{The Limitation of $\necessary$/$\possible$}

Standard modal logic has:
\begin{itemize}
    \item $\necessary\varphi$ = ``all accessible worlds satisfy $\varphi$'' ($\forall$)
    \item $\possible\varphi$ = ``some accessible world satisfies $\varphi$'' ($\exists$)
\end{itemize}

But natural reasoning uses many more quantifiers:
\begin{itemize}
    \item ``\emph{Most} options are safe''
    \item ``\emph{At least 3} escape routes exist''
    \item ``\emph{Exactly one} successor state''
    \item ``With \emph{probability $\geq 0.9$}, success''
\end{itemize}

\begin{warning}
``At least 3 safe options'' is very different from ``some safe option.'' Standard modal logic can't distinguish them.
\end{warning}

\section{Graded Modalities}

\begin{definition}[Graded Modal Logic]
Extend the language with counting modalities:
\begin{align*}
\possible_{\geq n}\varphi &\quad \text{``at least $n$ accessible worlds satisfy $\varphi$''} \\
\possible_{\leq n}\varphi &\quad \text{``at most $n$ accessible worlds satisfy $\varphi$''} \\
\possible_{= n}\varphi &\quad \text{``exactly $n$ accessible worlds satisfy $\varphi$''}
\end{align*}
\end{definition}

\begin{definition}[Semantics]
\[
\mathcal{M}, w \models \possible_{\geq n}\varphi \iff |\{v : wRv \text{ and } \mathcal{M}, v \models \varphi\}| \geq n
\]
\end{definition}

Note: $\possible\varphi \equiv \possible_{\geq 1}\varphi$ and $\necessary\varphi \equiv \possible_{\leq 0}\neg\varphi$.

\section{Majority Quantifier}

\begin{definition}[Majority Modality]
\[
\mathsf{M}\varphi \quad \text{``most accessible worlds satisfy $\varphi$''}
\]
Semantics:
\[
\mathcal{M}, w \models \mathsf{M}\varphi \iff |\{v : wRv \land v \models \varphi\}| > |\{v : wRv \land v \not\models \varphi\}|
\]
\end{definition}

\begin{keyinsight}
The majority quantifier $\mathsf{M}$ cannot be defined using $\forall$/$\exists$ (Lindström). It's genuinely more expressive.
\end{keyinsight}

\section{Probabilistic Modality}

\begin{definition}[Probabilistic Modal Logic]
Given a probability distribution $\mu_w$ over successors of $w$:
\[
\mathcal{M}, w \models P_{\geq r}\varphi \iff \mu_w(\{v : wRv \land v \models \varphi\}) \geq r
\]
\end{definition}

This requires more structure: not just accessibility $R$, but a probability measure.

\section{Meta-Theoretic Trade-offs}

Adding quantifiers has costs:

\begin{center}
\begin{tabular}{lll}
\textbf{Property} & \textbf{Basic Modal} & \textbf{Graded/Majority} \\
\hline
Compactness & \checkmark & Often fails \\
Finite model property & \checkmark & Often fails \\
Decidability & PSPACE & Higher complexity \\
\end{tabular}
\end{center}

\begin{theorem}[Lindström]
First-order logic is the unique logic with compactness and Löwenheim-Skolem. Adding quantifiers breaks at least one.
\end{theorem}

\section{Connection to Description Logic}

Description logics (used in OWL, knowledge graphs) have number restrictions:
\begin{align*}
(\geq 3\ \mathsf{hasChild}.\mathsf{Doctor}) &\quad \text{``at least 3 children are doctors''} \\
(\leq 1\ \mathsf{hasSpouse}) &\quad \text{``at most 1 spouse''}
\end{align*}

These are graded modalities in disguise!

\section{Why This Matters}

For AI agents:
\begin{itemize}
    \item ``Ensure at least 2 fallback options'' (robustness)
    \item ``Most paths lead to success'' (probabilistic planning)
    \item ``Exactly one next state'' (determinism)
\end{itemize}

Standard modal logic can't express these. Graded modal logic can.
