% Section: The Ancient World

\subsection{What Logic Is (and Isn't)}

In everyday speech, ``logic'' means something like ``rational thinking.'' We say ``that's not logical'' when someone's argument doesn't make sense.

\textbf{Formal logic is not this.}

Formal logic is not \emph{only} about being smart or avoiding fallacies---those may be corollaries, but they are not the core. The core is something much more specific:

\begin{quote}
Logic is the study of \textbf{implication}---the ``if... then...'' structure.

When can we say that one thing \emph{necessarily follows} from another?
\end{quote}

This is where logic \emph{started}. But as we will see, the subject has traveled far from this origin---not abandoning it, but generalizing it beyond recognition.

\subsection{Athens: Where It Began}

Why would anyone study such an abstract thing?

The answer lies in ancient Athens, where democracy meant public debate. To win an argument, you needed to show that your conclusion \textbf{necessarily follows} from premises your opponent already accepts.

\begin{history}
The word ``logic'' comes from the Greek \emph{logos}, which means both ``word'' and ``reason.'' For the Greeks, language and thought were intimately connected.
\end{history}

\begin{definition}[Validity, informally]
An argument is \textbf{valid} if the conclusion necessarily follows from the premises. That is, \emph{if} the premises are true, the conclusion \emph{must} be true.
\end{definition}

Validity is purely about the \textbf{structure} of the argument, not its content.

\subsection{Plato's World of Forms}

Before Aristotle systematized logic, his teacher Plato asked: \textbf{what is truth?}

Plato observed that the physical world is messy and changeable. Yet we have concepts of \emph{perfect} circles, \emph{ideal} chairs, \emph{just} societies.

\begin{history}
Plato's answer: there is a realm of \textbf{Forms}---eternal, unchanging, perfect archetypes. The physical world is just a shadow of this realm.
\end{history}

\begin{intuition}
Logical truths are not about this table or that chair. They are about the \emph{form} of arguments themselves. When we prove ``if all A are B, and all B are C, then all A are C,'' we are talking about \textbf{structure}, not any particular A, B, or C.

In a sense, logic lives in Plato's realm of Forms.
\end{intuition}

\subsection{Aristotle's Syllogism}

Plato's student Aristotle wanted to \textbf{systematize} valid reasoning---to give rules that anyone could follow to check if an argument is valid.

The result was the \textbf{syllogism}, the first formal system in history.

\begin{example}[A syllogism]
\begin{inference}
All men are mortal. \quad (major premise)\\
Socrates is a man. \quad (minor premise)\\
\rule{5cm}{0.4pt}\\
Therefore, Socrates is mortal. \quad (conclusion)
\end{inference}
\end{example}

The key insight: validity depends only on \textbf{form}, not content. We can replace the terms:

\begin{inference}
All A are B.\\
S is an A.\\
\rule{4cm}{0.4pt}\\
Therefore, S is B.
\end{inference}

Aristotle catalogued all valid forms and showed how to reduce complex arguments to these basic patterns.

\begin{history}[title={The Medieval Mnemonics}]
Medieval scholars labeled the four proposition types with vowels (A, E, I, O from \emph{affirmo} and \emph{nego}), then named each valid syllogism so its vowels encode the structure:
\begin{itemize}
    \item \textbf{Barbara} (AAA): All M are P. All S are M. $\therefore$ All S are P.
    \item \textbf{Celarent} (EAE): No M are P. All S are M. $\therefore$ No S are P.
\end{itemize}
This is perhaps the first example of \textbf{encoding logic in notation}.
\end{history}

\begin{keyinsight}
Aristotle's revolution: \textbf{form can be studied independently of content}.

This is the founding insight of all formal logic.
\end{keyinsight}
