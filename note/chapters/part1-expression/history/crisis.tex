% Section: The Foundational Crisis

Frege's dream was to derive mathematics from logic. In 1903, just as his \emph{Grundgesetze} was going to press, Russell found a contradiction.

\begin{history}[title={Russell's Paradox}]
Consider the set $R = \{ x \mid x \notin x \}$---all sets that don't contain themselves.

Does $R$ contain itself?
\begin{itemize}
    \item If $R \in R$, then by definition $R \notin R$. Contradiction.
    \item If $R \notin R$, then by definition $R \in R$. Contradiction.
\end{itemize}
\end{history}

Frege's system was inconsistent. Russell and Whitehead spent a decade fixing it in \emph{Principia Mathematica} (1910--1913), introducing type theory.

\begin{intuition}
The crisis revealed: logic is powerful but \textbf{dangerous}. The same expressive power that lets you talk about ``all sets'' also lets you construct paradoxes.
\end{intuition}
