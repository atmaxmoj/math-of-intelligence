% Section: The Birth of Modern Logic

\subsection{Leibniz's Dream}

In the 17th century, Leibniz had a vision:

\begin{history}
Leibniz dreamed of a \emph{characteristica universalis}: a universal symbolic language for all human thought. And a \emph{calculus ratiocinator}: a mechanical method to compute truth.

``When there are disputes among persons, we can simply say: let us calculate, and see who is right.''
\end{history}

Leibniz never achieved it. But his dream planted a seed: \textbf{reasoning as calculation}.

\subsection{Boole's Algebra}

Two centuries later, George Boole took the first real step.

\begin{history}[title={The Laws of Thought, 1854}]
Boole realized that AND, OR, NOT could be treated as algebraic operations. Propositions became variables, connectives became operations, and logical reasoning became solving equations.
\end{history}

\begin{example}[Boolean algebra]
Let $p$ = ``it is raining'' and $q$ = ``the ground is wet.''

\begin{center}
\begin{tabular}{cc|cccc}
$p$ & $q$ & $\neg p$ & $p \land q$ & $p \lor q$ & $p \to q$ \\
\hline
T & T & F & T & T & T \\
T & F & F & F & T & F \\
F & T & T & F & T & T \\
F & F & T & F & F & T \\
\end{tabular}
\end{center}

Boole showed these satisfy algebraic laws: $p \land q = q \land p$, $p \lor (q \land r) = (p \lor q) \land (p \lor r)$, etc.
\end{example}

But Boole's algebra could only express \textbf{propositional} logic. ``All men are mortal'' is just a single symbol $p$---the internal structure is invisible.

\subsection{Frege's Revolution}

The real transformation came in 1879, when Gottlob Frege published \emph{Begriffsschrift} (Concept-Script).

Frege wanted to derive mathematics from pure logic. To do this, he introduced:

\begin{keyinsight}
\begin{enumerate}
    \item \textbf{Quantifiers}: $\forall x$ (``for all $x$'') and $\exists x$ (``there exists $x$'')
    \item \textbf{Predicates and relations}: $P(x)$, $R(x, y)$ instead of ``$S$ is $P$''
    \item \textbf{Nested structure}: $\forall x \exists y \, R(x, y)$ differs from $\exists y \forall x \, R(x, y)$
\end{enumerate}
\end{keyinsight}

\begin{intuition}
$\forall$ and $\exists$ are the \textbf{minimal pair}, corresponding to Aristotle's ``All'' and ``Some.'' They are duals:
\[
\forall x \, \varphi \;\equiv\; \neg \exists x \, \neg \varphi
\]
Later, modal logic will use the same pattern: $\Box$ = ``in all worlds,'' $\Diamond$ = ``in some world.'' This is not a coincidence.
\end{intuition}

This is \textbf{first-order logic}. Frege's original system was actually higher-order (quantifying over predicates), which led to Russell's paradox.

\subsubsection{The Semantics of First-Order Logic}

Propositional logic has truth tables. What is the analogous ``truth table'' for first-order logic?

The answer is a \textbf{structure} (or \textbf{model}):

\begin{definition}[Structure]
A structure $\mathcal{M}$ consists of:
\begin{enumerate}
    \item A non-empty set $D$, the \textbf{domain}
    \item For each constant $c$, an element $c^\mathcal{M} \in D$
    \item For each predicate $P$, a subset $P^\mathcal{M} \subseteq D^n$
    \item For each function $f$, a function $f^\mathcal{M}: D^n \to D$
\end{enumerate}
\end{definition}

\begin{example}[Arithmetic]
Structure $\mathcal{N}$: domain $D = \{0, 1, 2, \ldots\}$, $0^\mathcal{N} = 0$, $s^\mathcal{N}(n) = n+1$, $<^\mathcal{N} = \{(m,n) \mid m < n\}$.

In $\mathcal{N}$, the formula $\forall x \, (x < s(x))$ is \textbf{true}.

But in a different structure with $D = \{a, b\}$ and $s(b) = a$, the same formula is \textbf{false}.
\end{example}

The key insight: \textbf{the same formula can be true in one structure and false in another}. Truth is always relative to a structure.

The formal satisfaction relation $\mathcal{M} \models \varphi$ is defined recursively:
\begin{center}
\begin{tabular}{ll}
$\mathcal{M} \models P(t_1, \ldots, t_n)$ & iff $(t_1^\mathcal{M}, \ldots, t_n^\mathcal{M}) \in P^\mathcal{M}$ \\[4pt]
$\mathcal{M} \models \neg \varphi$ & iff $\mathcal{M} \not\models \varphi$ \\[4pt]
$\mathcal{M} \models \varphi \land \psi$ & iff $\mathcal{M} \models \varphi$ and $\mathcal{M} \models \psi$ \\[4pt]
$\mathcal{M} \models \forall x \, \varphi$ & iff for \textbf{every} $d \in D$: $\mathcal{M} \models \varphi[x \mapsto d]$
\end{tabular}
\end{center}
