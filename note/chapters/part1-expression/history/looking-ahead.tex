% Section: Looking Ahead

\begin{keyinsight}
Logic studies \textbf{the invariance of forms}.

What properties are preserved under what transformations?
\begin{itemize}
    \item Valid inference preserves truth
    \item Bisimulation preserves modal properties
    \item Computable functions preserve finiteness of description
    \item Homomorphisms preserve algebraic structure
\end{itemize}
\end{keyinsight}

This is why logic serves so many fields: philosophers analyzing concepts, mathematicians studying foundations, linguists uncovering semantic structure, computer scientists verifying programs.

A question lingers: can we study ``invariance'' itself as a mathematical object? The answer is \textbf{category theory}---the subject of the next chapter.

\begin{summary}
\begin{itemize}
    \item Logic began as the study of valid inference (Aristotle's syllogism)
    \item Medieval logic was sophisticated but forgotten after the Humanist turn
    \item Modern logic arose from mathematics' need for quantifiers (Frege)
    \item The syntax/semantics distinction ($\vdash$ vs $\models$) emerged from foundational debates
    \item Three schools---logicism, formalism, intuitionism---each gave different answers
    \item Key properties: soundness and consistency (non-negotiable); completeness and decidability (desirable)
    \item The foundational crisis led to four pillars: proof theory, model theory, computability, set theory
    \item Core insight: logic studies the \textbf{invariance of forms}
\end{itemize}
\end{summary}
