% Section: The Medieval Interlude

In 1781, Kant declared that logic ``has not been able to advance a single step'' since Aristotle.\footnote{Kant, \emph{Critique of Pure Reason}, Bviii.}

This was largely false.

\subsection{The Medieval Golden Age}

The historian J.M. Bocheński identified \textbf{three} golden periods of logic: ancient Greece, the medieval scholastic period, and the mathematical period of the 19th--20th centuries.\footnote{See Catarina Dutilh Novaes, ``The Rise and Fall and Rise of Logic,'' \emph{Aeon} (2017).}

Medieval logicians---William of Ockham, Jean Buridan, Walter Burley---developed sophisticated systems beyond Aristotle:

\begin{itemize}
    \item \textbf{Supposition theory}: A proto-semantics distinguishing how terms refer in different contexts---remarkably close to modern use/mention distinctions.
    \item \textbf{Consequentiae}: A theory of logical consequence beyond the syllogism.
    \item \textbf{Insolubilia}: The study of semantic paradoxes like the Liar, centuries before Russell and Tarski.
    \item \textbf{Obligationes}: Formalized disputation games---an early dialogue logic.
\end{itemize}

\subsection{The Humanist Catastrophe}

Around 1530, Renaissance humanists swept away the scholastic tradition. They found the ``barbarous language and twisted Latin of the scholastics'' distasteful, preferring Cicero's elegance to Ockham's precision.

\begin{quote}
``After about 1530 not only did new writing on the specifically medieval contributions to logic cease, but the publication of medieval logicians virtually ceased.''
\end{quote}

The medieval advances were not refuted---they were simply \emph{forgotten}. It was only in the late 20th century that historians rediscovered what had been lost.

\subsection{What Actually Changed}

So why didn't medieval logic break through to the modern form?

Medieval logic served philosophy and theology. Its paradigm cases were ``All men are mortal'' and ``God is good''---these fit the subject-predicate structure of syllogisms.

But in the 17th century, a new customer appeared: \textbf{mathematics}. And mathematics needed to say things syllogisms could not express:

\begin{itemize}
    \item ``For every $\epsilon > 0$, there exists a $\delta > 0$ such that...'' (limits)
    \item ``There exists a unique $x$ such that $f(x) = 0$'' (existence and uniqueness)
    \item ``For all $n$, if $P(n)$ then $P(n+1)$'' (induction)
\end{itemize}

These involve \textbf{nested quantifiers}---alternations of ``for all'' and ``there exists.'' Aristotle's ``All A are B'' cannot express ``for every $\epsilon$ there exists a $\delta$.''

The real revolution came when mathematicians needed to formalize their own reasoning.
