\chapter{How Much Can We Express?}

\begin{goals}
\begin{itemize}
    \item Understand expressiveness as a measure of logical power
    \item Compare modal logic to first-order logic
    \item Introduce correspondence theory
\end{itemize}
\end{goals}

\section{The Standard Translation}

\begin{definition}[Standard Translation]
Define $\mathrm{ST}_x : \mathcal{L}(\necessary) \to \mathcal{L}_{\mathrm{FO}}$:
\begin{align*}
    \mathrm{ST}_x(p) &= P(x) \\
    \mathrm{ST}_x(\neg\varphi) &= \neg\mathrm{ST}_x(\varphi) \\
    \mathrm{ST}_x(\necessary\varphi) &= \forall y (R(x,y) \to \mathrm{ST}_y(\varphi))
\end{align*}
\end{definition}

Modal logic is a \emph{fragment} of first-order logic. But which fragment?

\section{The van Benthem Characterization}

\begin{theorem}[van Benthem]
A first-order formula $\varphi(x)$ is equivalent to a modal formula iff it is invariant under bisimulation.
\end{theorem}

\begin{keyinsight}
Modal logic = bisimulation-invariant fragment of first-order logic.
\end{keyinsight}

\section{Frame Correspondence}

Some modal formulas correspond to first-order conditions on frames:

\begin{center}
\begin{tabular}{ll}
\textbf{Modal Axiom} & \textbf{Frame Condition} \\
\hline
$\necessary p \to p$ & Reflexive \\
$\necessary p \to \necessary\necessary p$ & Transitive \\
$p \to \necessary\possible p$ & Symmetric \\
\end{tabular}
\end{center}

% TODO: Sahlqvist theorem
