\documentclass[11pt,a4paper]{book}

% Encoding and fonts
\usepackage[utf8]{inputenc}
\usepackage[T1]{fontenc}
\usepackage{lmodern}

% Math packages
\usepackage{amsmath,amssymb,amsthm}
\usepackage{mathtools}
\usepackage{bussproofs}  % for proof trees

% Logic symbols
\usepackage{stmaryrd}    % \llbracket, \rrbracket
\newcommand{\necessary}{\Box}
\newcommand{\possible}{\Diamond}
\newcommand{\forces}{\Vdash}
\newcommand{\nforces}{\nVdash}
\newcommand{\bisim}{\underline{\leftrightarrow}}

% Theorem environments
\theoremstyle{definition}
\newtheorem{definition}{Definition}[chapter]
\newtheorem{example}[definition]{Example}

\theoremstyle{plain}
\newtheorem{theorem}[definition]{Theorem}
\newtheorem{lemma}[definition]{Lemma}
\newtheorem{proposition}[definition]{Proposition}
\newtheorem{corollary}[definition]{Corollary}

\theoremstyle{remark}
\newtheorem{remark}[definition]{Remark}

% TikZ for Kripke frames
\usepackage{tikz}
\usetikzlibrary{arrows.meta, positioning, shapes}

% Hyperlinks
\usepackage{hyperref}
\hypersetup{
    colorlinks=true,
    linkcolor=blue,
    citecolor=blue,
    urlcolor=blue
}

% Bibliography
\usepackage[style=authoryear,backend=biber]{biblatex}
% \addbibresource{references.bib}

% Title
\title{Modal Logic: A Comprehensive Treatment}
\author{Sijie Wang}
\date{\today}

\begin{document}

\frontmatter
\maketitle
\tableofcontents

\mainmatter

\chapter{Introduction}

Modal logic extends classical propositional logic with operators expressing modality---notions such as possibility, necessity, knowledge, belief, obligation, and time.

\section{Historical Background}

The modern study of modal logic begins with C.I. Lewis's work in the early 20th century...

\section{Why Modal Logic?}

% TODO: motivation and applications

\chapter{Syntax and Semantics}

\section{Modal Languages}

\begin{definition}[Modal Language]
Let $\Phi$ be a countable set of propositional variables. The \emph{modal language} $\mathcal{L}(\necessary, \possible)$ is defined by the grammar:
\[
\varphi ::= p \mid \neg\varphi \mid (\varphi \land \psi) \mid \necessary\varphi
\]
where $p \in \Phi$. We define $\possible\varphi := \neg\necessary\neg\varphi$.
\end{definition}

\section{Kripke Semantics}

\begin{definition}[Kripke Frame]
A \emph{Kripke frame} is a pair $\mathcal{F} = (W, R)$ where:
\begin{itemize}
    \item $W$ is a non-empty set of \emph{possible worlds}
    \item $R \subseteq W \times W$ is a binary \emph{accessibility relation}
\end{itemize}
\end{definition}

\begin{definition}[Kripke Model]
A \emph{Kripke model} is a triple $\mathcal{M} = (W, R, V)$ where:
\begin{itemize}
    \item $(W, R)$ is a Kripke frame
    \item $V : \Phi \to \mathcal{P}(W)$ is a \emph{valuation function}
\end{itemize}
\end{definition}

\begin{definition}[Satisfaction]
Let $\mathcal{M} = (W, R, V)$ be a Kripke model and $w \in W$. The satisfaction relation $\mathcal{M}, w \models \varphi$ is defined inductively:
\begin{align*}
\mathcal{M}, w &\models p &&\text{iff } w \in V(p) \\
\mathcal{M}, w &\models \neg\varphi &&\text{iff } \mathcal{M}, w \not\models \varphi \\
\mathcal{M}, w &\models \varphi \land \psi &&\text{iff } \mathcal{M}, w \models \varphi \text{ and } \mathcal{M}, w \models \psi \\
\mathcal{M}, w &\models \necessary\varphi &&\text{iff for all } v \in W: wRv \Rightarrow \mathcal{M}, v \models \varphi
\end{align*}
\end{definition}

\begin{example}
Consider the frame with $W = \{w_0, w_1, w_2\}$ and $R = \{(w_0, w_1), (w_0, w_2), (w_1, w_1)\}$:

\begin{center}
\begin{tikzpicture}[
    world/.style={circle, draw, minimum size=8mm},
    arrow/.style={-{Stealth}, thick}
]
\node[world] (w0) at (0,0) {$w_0$};
\node[world] (w1) at (2,1) {$w_1$};
\node[world] (w2) at (2,-1) {$w_2$};

\draw[arrow] (w0) -- (w1);
\draw[arrow] (w0) -- (w2);
\draw[arrow] (w1) to [loop right] (w1);
\end{tikzpicture}
\end{center}

% TODO: complete the example
\end{example}

\section{Validity and Consequence}

% TODO

\chapter{Bisimulation and Model Theory}

\section{Bisimulation}

\begin{definition}[Bisimulation]
Let $\mathcal{M} = (W, R, V)$ and $\mathcal{M}' = (W', R', V')$ be Kripke models. A non-empty relation $Z \subseteq W \times W'$ is a \emph{bisimulation} between $\mathcal{M}$ and $\mathcal{M}'$ if whenever $wZw'$:
\begin{enumerate}
    \item (Atomic harmony) For all $p \in \Phi$: $w \in V(p) \Leftrightarrow w' \in V'(p)$
    \item (Zig) If $wRv$, then there exists $v' \in W'$ such that $w'R'v'$ and $vZv'$
    \item (Zag) If $w'R'v'$, then there exists $v \in W$ such that $wRv$ and $vZv'$
\end{enumerate}
\end{definition}

\begin{theorem}[Bisimulation Invariance]
If $Z$ is a bisimulation between $\mathcal{M}$ and $\mathcal{M}'$ and $wZw'$, then for all modal formulas $\varphi$:
\[
\mathcal{M}, w \models \varphi \iff \mathcal{M}', w' \models \varphi
\]
\end{theorem}

\begin{proof}
By induction on the structure of $\varphi$.
% TODO: complete proof
\end{proof}

\section{Bounded Morphisms}

% TODO

\section{Generated Submodels}

% TODO

\chapter{Soundness and Completeness}

\section{Normal Modal Logics}

% TODO

\section{Canonical Models}

% TODO

\section{Completeness Theorems}

% TODO

\chapter{Correspondence Theory}

\section{Frame Definability}

% TODO

\section{The Sahlqvist Theorem}

% TODO

\chapter{Decidability}

\section{Finite Model Property}

% TODO

\section{Filtration}

% TODO

\chapter{Combining Modal Logics}

\section{Fusion}

% TODO

\section{Products}

% TODO

\backmatter

% \printbibliography

\end{document}
